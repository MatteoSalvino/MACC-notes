\chapter{Framework Android}
Adesso introdurremo delle definizioni di alcuni componenti presenti nel framework Android, come:
\begin{itemize}
\item \textbf{Drawable}: esso è un'astrazione generale per qualcosa che può essere disegnato sullo schermo.
\item \textbf{Vista}: essa a differenza del precedente componente è anche in grado di gestire gli eventi.
\item \textbf{Immagine digitale}: un'immagine digitale W x H è una matrice di pixels (Width, Height). Un pixel è la più piccola unità visuale che può essere controllata. Quando si parla di dimensione ci riferiamo al numero totale di pixels. Inoltre, spesso viene definito l'aspect ratio W/H.
\item \textbf{Immagine vettoriale}: essa descrive un' immagine in termini di punti, i quali sono connessi da linee e curve per formare poligoni ed altre forme. I formati più popolari sono: PDF, SVG e EPS.
\item \textbf{Schermo}: Uno schermo ha una superficie X x Y, una diagonale $\sqrt[2]{X^2 + Y^2}$, e un aspect ratio $AR = Y:X$, dove Y è il numero più grande. Spesso le dimensioni degli schermi vengono fornite come la lunghezza della diagonale in inches (1 in = 2.54 cm). E' possibile determinare anche quanti pixel sono presenti per inch (PPI).
\end{itemize}
In generale fornita un'immagine W x H, non possiamo dire nulla sulla sua risoluzione spaziale, questo perchè tale risoluzione dipende dalle dimensioni dello schermo X x Y dove verrà mostrata l'immagine. Ha senso incrementare sempre di più il PPI ? L'occhio umano ha la sua massima risoluzione (300 PPI), così non ha senso andare oltre il limite. Infatti schermi con elevati PPI mappano un pixel software su diversi pixels hardware.
\section{Il sistema delle viste}
Esso è un sistema per organizzare l'interfaccia grafica. \textbf{ViewGroup} è una vista che contiene altre viste (ad esempio il Linear Layout). Un \textbf{container} è un altro tipo di ViewGroup utilizzato per ospitare del contenuto dinamico, oppure del contenuto che non può essere adattato alle dimensioni dello schermo (supporto dello scrolling). La gestione del contenuto dinamico richiede un'implementazione precisa per ottimizzare le prestazioni: il contenuto dovrebbe essere recuperato da uno sorgente lenta (internet), utilizzare un thread separato per eseguire il download di contenuti, gli elementi della vista devono essere aggiunti e aggiornare la vista. Gli \textbf{stili} e i \textbf{temi} su Android ci consentono di separare i dettagli della progettazione dell'applicazione dalla struttura e dal comportamento dell'interfaccia utente. Uno \textbf{stile} è una collezione di attributi che specificano l'apparenza per una singola vista. Uno stile può specificare attributi come il colore e la dimensione del font, il colore di sfondo, etc. Un \textbf{tema} è un tipo di stile che viene applicato all'intera applicazione, non solo ad una singola vista.
\section{Activity}
Un'applicazione è costituita da almeno un'activity, la quale è controllata dalla GUI nello schermo. Essa viene eseguita all'interno del thread principale e dovrebbe reagire all'input dell'utente (gestita dal gestore delle Activity). Le activity nel sistema sono gestite attraverso una pila posteriore (back stack). In particolare, quando un activity viene creata, essa viene posizionata in cima a tale pila e diventa attiva. Le activity precedenti rimangono nella pila in attesa della distruzione dell'activity attualmente attiva. L'insieme delle activity lanciate da un utente è chiamato \textbf{Task}. Tutte le componenti dell'applicazione vengono eseguite all'interno del thread principale. Tuttavia, possono essere creati altri thread per eseguire delle operazioni che richiedono molto tempo. \\Ogni activity viene creata dal launcher quando viene selezionata l'icona corrispondente. Essa può essere creata anche da un'altra activity oppure da un altro componente software. Un activity può essere in tre differenti stati: esecuzione, pausa e arrestata. Un activity in esecuzione è in primo piano e ha il focus, così l'utente può interagire con essa. Un activity in pausa è parzialmente visibile ma non ha il focus. Questo si può verificare quando un'altra activity viene eseguita ma non ricopre l'intero schermo. Un activity è arrestata quando viene completamente oscurata da un'altra. Le activity in pausa e arrestate possono essere terminate dal sistema quando ha bisogno delle risorse. Quando un activity viene creata, viene chiamato il metodo onCreate(). Esso inizializza tutte le viste, i dati, etc, ed è seguito dal metodo onStart(). Questo metodo è chiamato prima che l'activity diventi visibile e quando l'activity viene riavviata dopo lo stato di arresto (onRestart() -> onStart()). Il metodo onResume() viene chiamato dopo il metodo onStart() oppure quando l'activity viene ripristinata dallo stato di pausa. Il metodo onPause() viene chiamato prima del ripristino di un'altra activity. Esso è utilizzato per eseguire il commit di cambiamenti non salvati e per arrestare i threads, le animazioni, etc. Esso deve essere veloce perchè l'altra activity non può essere ripristinata fino a quando l'activity corrente non abbia terminato i processi sopra citati. Il metodo onStop() è chiamato quando l'activity non è più visibile all'utente. Se l'activity viene ripristinata, questo metodo è seguito da onRestart(), altrimenti se l'activity deve essere distrutta viene chiamato il metodo onDestroy(). Per memorizzare un piccolo ammontare di dati che possono essere serializzati e deserializzati può essere utilizzato un \textbf{Bundle}. Tuttavia, esso diventa inefficiente al crescere dell'ammontare dei dati da memorizzare. Android risolve tale problema, fornendo un modo per memorizzare i dati attraverso il \textbf{ViewModel}. Un activity è creata se è il target di un messaggio speciale chiamato \textbf{Intent}.
\subsection{Intent}
Un intent è un oggetto di messaggistica che può essere utilizzato per richiedere un'azione da un'altro componente dell'applicazione. Esso facilita la comunicazione fra le componenti, ma può essere anche utilizzato per eseguire un'activity o un servizio. Gli intent consentono di chiamare un'activity ed ottenere il risultato. L'activity chiamante non attenderà in modo sincrono il risultato, ma l'activity chiamata quando avrà elaborato il risultato chiamerà il metodo setResult, il quale eseguirà il metodo onActivityResult dell'activity chiamante. Quando un'activity vuole passare dei dati ad un'altra activity, ha bisogno di settare dei campi extra nell'intent. Ad esempio, possiamo raggiungere questo scopo utilizzando un bundle, il quale può inglobare tipi di dato primitivi e strutture dati. Un intent viene classificato in:
\begin{itemize}
\item \textbf{Unicast}: intent intesi solamente per uno specifico componente oppure per una determinata azione da compiere. Gli intent di questo tipo si dividono in:
\begin{itemize}
\item \textbf{Espliciti}: quando il target dell'intent è dichiarato all'interno di esso.
\item \textbf{Impliciti}: quando l'intent specifica solamente l'azione che il componente dovrebbe compiere. Se sono disponibili diverse activity che possono eseguire l'azione richiesta, allora l'utente ha bisogno di selezionarne una.
\end{itemize}
\item \textbf{Broadcast}: intent che comunica delle informazioni a tutti i componenti (vedi Broadcast Receivers).
\item \textbf{Pending}: intent con i quali possono essere attivati dei componenti in futuro (utilizzato per le notifiche).
\end{itemize}
Un \textbf{intent-filter} è un'espressione nel manifesto dell'applicazione che specifica il tipo di intent che il componente vorrebbe ricevere. Dichiarando un intent filter per un activity, rendiamo possibile ad altre applicazioni di lanciare direttamente tale activity con un determinato tipo di intent. Ad esempio, supponiamo che l'activity A nel package PA vuole chiamare l'activity B nel package PB. L'activity B definisce un intent-filter contenente un'azione personalizzata e la categoria di default. L'activity A crea un intent con l'azione personalizzata e chiama startActivity. Il sistema trova il componente appropriato da eseguire confrontando il contenuto dell'intent con il contenuto dell'intent filter dichiarato nel manifesto di altre applicazioni sul dispositivo. Le activity vengono eseguite come due utenti separati.\\Un \textbf{Broadcast Receiver} è un componente che ci consente di registrare gli eventi di un'applicazione o del sistema. Tutti i ricevitori registrati per un evento vengono notificati da Android una volta che l'evento si verifica. La parte di implementazione della registrazione consiste nella creazione di un intent filter per indicare gli specifici intent broadcast cui il ricevitore deve ascoltare, facendo riferimento alla stringa dell'azione dell'intent broadcast. Quando viene trovata una corrispondenza broadcast, il metodo onReceive() del ricevitore viene chiamato; a questo punto il metodo ha a disposizione 5 secondi per eseguire le attività necessarie prima di restituire il controllo. I ricevitori broadcast non hanno bisogno di essere eseguiti di continuo. Gli intent broadcast sono degli oggetti intent che diventano broadcast attraverso una chiamata al metodo sendBroadcast() della classe Activity(). Inoltre, per fornire un sistema di comunicazione tra le componenti dell'applicazione, gli intent broadcast vengono anche utilizzati dal sistema Android per notificare le applicazioni interessate per quanto riguarda gli eventi principali del sistema (ad esempio la connessione/disconnessione del cavo di alimentazione).
\section{Permessi}
Android è un sistema operativo a privilegi separati, nel quale ogni applicazione viene eseguita con una diversa identità di sistema (Linux user id e group id). Le parti del sistema sono anche separate in diverse identità. In tal modo, Linux separa le applicazioni l'una dalle altre e dal sistema. Visto che ogni applicazione Android opera in un processo sandbox, le applicazioni devono richiedere esplicitamente i permessi a loro necessari per ulteriori capacità non fornite nella sandbox nel manifesto. Dipende da quanto l'area è sensibile, il sistema potrebbe consentire i permessi automaticamente oppure potrebbe chiedere all'utente se approvare oppure rifiutare la richiesta. Quando un permesso viene concesso per un' applicazione, il sistema assegnerà un Group id supplementare corrispondente al permesso. Se la piattaforma del dispositivo è Android 6.0 (livello API 23) o superiore, e il targetSdkVersion dell'applicazione è 23 o superiore, l'applicazione richiederà i permessi all'utente a tempo di esecuzione. Un utente può revocare i permessi in qualsiasi momento, così l'applicazione ha bisogno di controllare se ha i permessi ogni volta che viene eseguita. Se la piattaforma del dispositivo è Android 5.1 (livello API 22) o inferiore, oppure se il targetSdkVersion è 22 o inferiore, il sistema chiederà di fornire i permessi quando l'utente installerà l'applicazione. Se aggiungiamo un nuovo permesso ad una versione aggiornata dell'applicazione, il sistema chiederà all'utente di consentire tale permesso quando l'utente aggiornerà l'applicazione.
\section{Servizi}
Un servizio è un componente dell'applicazione che può eseguire delle lunghe operazioni in background, il quale non fornisce alcuna interfaccia utente. Il ciclo di vita dei servizi è indipendente dal componente che lo ha lanciato, così loro possono continuare la loro esecuzione anche se tale componente ha terminato la sua esecuzione. I servizi si dividono in due categorie:
\begin{itemize}
\item \textbf{Foreground}: un servizio di questo tipo esegue alcune operazioni che sono visibile all'utente. Per esempio, un'applicazione per la musica potrebbe utilizzare un servizio di foreground per procedere con l'esecuzione di un brano. Essi devono mostrare un'icona nella barra di stato del dispositivo, invocando tipicamente il metodo startForeground(id, notification). Il sistema difficilmente terminerà questo tipo di servizi.
\item \textbf{Background}: un servizio di background esegue un operazione non direttamente visibile all'utente. Per esempio, se un'applicazione utilizza un servizio per comprimere la memoria che utilizza, questo sarà probabilmente un servizio di background. Il sistema ha più probabilità di terminare un servizio di background rispetto a uno di foreground. A partire da Oreo, per ragioni di prestazioni il sistema impone delle restrizioni sui servizi di background in esecuzione quando l'applicazione stessa non è in foreground. Per eseguire delle operazioni di background viene raccomandato l'utilizzo dello \textbf{JobScheduler}.
\end{itemize}
I servizi possono assumere due comportamenti:
\begin{itemize}
\item \textbf{Started service}: esso esegue una singola operazione e non restituisce il risultato direttamente al chiamante. Il servizio può essere eseguito in background indefinitivamente, anche se il componente che lo ha lanciato viene distrutto. Loro vengono eseguiti chiamando la funzione startService(). Quando il servizio ha completato la sua attività, esso dovrebbe chiamare il metodo stopSelf(). Alternativamente, un servizio in esecuzione può essere arrestato chiamando il metodo stopService(). Il metodo onBind() deve restituire null.
\item \textbf{Bound service}: esso offre un'interfaccia client-server che consente ai componenti di interagire con il servizio, inviare richieste, ricevere risultati e persino fare questo fra processi con la comunicazione fra processi (IPC). Un servizio di questo tipo verrà eseguito fino a quando il componente a cui è associato continuerà l'esecuzione. Loro vengono lanciati con una chiamata al metodo bindService(), ma permettono delle interazioni con il componente che li ha lanciati. Diversi componenti possono essere collegati ad uno stesso servizio simultaneamente. Per dissociarsi da un servizio, il componente chiama il metodo unbindService(). Quando tutti i componenti collegati ad un servizio vengono dissociati, il sistema lo termina. Un servizio di questo tipo può essere anche lanciato con una chiamata a startService() e poi i componenti possono associarsi ad esso invocando la funzione bindService(). In questo caso il servizio non verrà terminato quando tutti i componenti si saranno dissociati da esso.
\end{itemize}
Lo \textbf{JobScheduler} viene utilizzato quando un'attività non richiede un tempo esatto, ma potrebbe essere schedulato basato sulla combinazione dei requisiti dell'utente e del sistema. Per esempio, un'applicazione potrebbe aggiornare le notizie di mattina, ma potrebbe attende fino a quando il dispositivo sarà carico e connesso al wifi per aggiornarle, per preservare le risorse del sistema e i dati dell'utente. Tale classe permette di stabilire le condizioni oppure i parametri per eseguire una determinata attività. Inoltre, esso calcola il miglior istante temporale per schedulare l'esecuzione dell'attività richiesta. \\Un servizio viene eseguito sul thread principale, così se esso deve eseguire una lunga operazione dovrebbe creare un nuovo thread che esegue tale attività. \textbf{Intent Service} semplifica tale operazione. Intent Service è una sottoclasse di Service e predispone un thread per gestire le attività in background asincronamente. Inoltre, essa processa una richiesta alla volta in base alla politica FIFO. Una volta che il servizio ha gestito tutte le richieste, esso termina. Per processare una richiesta, il metodo onHandleIntent() viene chiamato, così il programmatore deve implementarlo. Visto che Intent Service è un servizio, deve essere necessariamente registrato nel manifesto dell'applicazione.
\section{Frammento}
Un frammento è una parte di un activity, il quale contribuisce all'interfaccia utente di tale activity. Un frammento può essere pensato come ad una sotto activity, mentre lo schermo completo con il quale l'utente interagisce è chiamato come activity. Un activity funge da collante per diversi frammenti. Un frammento ha il proprio layout e comportamento, così come un proprio ciclo di vita. Possiamo aggiungere oppure rimuovere dei frammenti in una singola activity per costruire un interfaccia utente multi riquadro. Un frammento può essere utilizzato in diverse activity. Il ciclo di vita di un frammento è strettamente in relazione con il ciclo di vita dell'activity ospitante. Ad esempio, quando un activity è messa in pausa, tutti i frammenti disponibili nell'activity verranno messi in pausa. I frammenti hanno delle callback extra rispetto alle activity, per gestire le interazioni con l'activity per eseguire azioni come la costruzione o distruzione dell'interfaccia utente del frammento. Questi metodi addizionali sono:
\begin{itemize}
\item onAttach() chiamato quando il frammento è stato associato con un activity.
\item onCreateView() chiamato per creare la gerarchia delle viste associata al frammento.
\item onActivityCreated() chiamato al completamento del metodo onCreate() dell'activity ospitante.
\item onDestroyView() chiamato quando la gerarchia delle viste associata al frammento viene rimossa.
\item onDetach() chiamato quando il frammento viene dissociato dall'activity.
\end{itemize} 
Android fornisce un manager chiamato \textbf{Fragment Manager} per aggiungere/rimuovere/trovare un frammento. Tipicamente, tale manager utilizza un \textbf{FrameLayout} per fornire l'area dove il frammento mostrerà la propria vista. Quando un frammento è creato programmaticamente, l'activity ospitante può utilizzare un Bundle per fornire dei dati al frammento. Diversi frammenti in una stessa activity possono comunicare utilizzando l'activity ospitante.
\section{RecyclerView}
E' una tipologia di vista che ci permette di mostrare una lista di dati. Le liste possono essere molto complesse, mostrare una serie di destinazioni, oppure dei dati inerenti a transazioni, etc. I benefici nell'utilizzare questa vista sono i seguenti:
\begin{itemize}
\item Per default, RecyclerView processa solamente gli elementi attualmente presenti sullo schermo. Quando l'utente effettua uno scrolling, tale componente si accorge che dei nuovi elementi devono essere mostrati ed effettua abbastanza lavoro per mostrarli.
\item Quando un elemento non è più visibile sullo schermo, la sua vista viene riciclata. Ciò significa che la sua vista verrà riutilizzata da un'altro elemento, aggiornando il rispettivo contenuto.
\item Quando un elemento cambia, invece di progettare l'intera lista, tale componente può aggiornare solamente tale elemento.
\end{itemize}
\section{Multithreading}
Quando viene eseguita l'activity principale di un'applicazione, essa viene eseguita sul thread principale. Questo è responsabile per lo smistamento degli eventi alla vista all'interno dello schermo. Il thread dell'interfaccia utente dovrebbe essere molto reattivo a gli input dell'utente, così lunghe operazioni non dovrebbe essere eseguite su di esso. Ci sono diversi modi per implementare il multithreading. Il primo è di utilizzare la programmazione standard dei thread, estendendo la classe Thread ed effettuando l'override del metodo run(). Se il thread di lavoro ha bisogno di aggiornare l'interfaccia utente, deve comunicare con il thread principale attraverso un meccanismo basato su messaggi: il thread principale crea un oggetto handler e il thread di lavoro lo utilizza per ottenere un messaggio vuoto e lo invia al thread principale. L'handler consente di inviare oggetti Message e Runnable. Un'altro modo per implementare il multithreading è utilizzare la classe AsyncTask. Essa semplifica l'interazione tra il thread principale e il thread di lavoro: essa consente di eseguire operazioni in background (metodo doInBackground()) e pubblicare i risultati nel thread principale (metodo onPostExecute()). Possiamo anche eseguire operazioni di background con Intent Services oppure utilizzando il framework Volley, il quale semplifica le richieste HTTP.
