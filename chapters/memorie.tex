\chapter{Memorie}
Dove memorizzare i dati di un'applicazione ? Abbiamo le seguenti possibilità:
\begin{itemize}
    \item \textbf{SharedPreference}: consente di memorizzare dati privati (booleani, float, int, long e string) come coppie chiave-valore. È una buona opzione per un piccolo ammontare di dati (es. stato di un gioco).
    \item \textbf{Preference}: è stato aggiunto in android X, per memorizzare le preferenze di un utente mostrate a quest'ultimo tramite le impostazioni dell'applicazione.
    \item \textbf{Memoria interna}: può essere utilizzata la memoria interna del dispositivo per memorizzare i dati di un utente. Tipicamente viene creata una cartella privata quando l'applicazione viene installata.
    \item \textbf{Memoria esterna}: può essere utilizzata la memoria esterna del dispositivo (SD) per memorizzare i dati di un utente, i quali sono accessibili da chiunque.
    \item \textbf{SQLite}: possiamo memorizzare i dati in un database con una determinata struttura dati. Viene raccomandato l'accesso attraverso la \textbf{Room Persistency Library}, con il quale la correttezza delle query verrà controllata a tempo di compilazione. Per accedere ai dati dell'applicazione utilizzando tale libreria, dobbiamo lavorare con i \textbf{Data Access Objects (DAO)}. Questo insieme di oggetti DAO costituisce il componente principale di Room, visto che ogni DAO include dei metodi che offrono un'accesso astratto al database dell'applicazione. Accedere al database utilizzando questi oggetti, ci permette di separare differenti componenti dell'architettura del database. Prima di tutto per utilizzare un DAO come interfaccia bisogna inserire l'annotazione @Dao. Per le operazioni CRUD verranno utilizzate le rispettive notazioni @Insert, @Query, @Update e @Delete.
    \item \textbf{Provider di contenuti}: esso è una repository di informazioni condivise fra diverse applicazioni. Fornisce un pieno controllo per le operazioni di lettura e scrittura, e un'interfaccia uniforme che può essere implementata in un file oppure in un database. Alcuni esempi di provider di contenuti sono Contatti, Calendario, etc. L'URI che viene utilizzato per richiedere una determinata informazione assume la seguente forma $content://authority/path/id$, dove \textbf{content} significa che vogliamo accedere al provider di contenuti, \textbf{authority} rappresenta il nome del provider, \textbf{path} è 0 oppure diversi segmenti che indicano i dati a cui vogliamo accedere e \textbf{id} specifica l'elemento. Possiamo utilizzare un provider di contenuti attraverso un intent con target il provider stesso. Inviare un intent all'applicazione Contatti del dispositivo ci consente di accedere al provider dei Contatti indirettamente. Infatti, l'intent lancia l'interfaccia utente dell'applicazione Contatti del dispositivo, nel quale un utente può effettuare delle operazioni relative ai Contatti. In questo modo l'utente può prendere un contatto dalla lista e restituirlo all'applicazione, modificare i dati di un contatto esistente, inserire un nuovo contatto e rimuovere un contatto.
\end{itemize}

La libreria \textbf{DataBinding} consente di associare dei valori nel layout ad un oggetto di dati. Essa consente di aggiungere dei listeners e notificare il cambiamento dei dati all'interfaccia utente. \textbf{Firebase} è un database in tempo reale. Molti database richiedono delle chiamate HTTP per ottenere e sincronizzare dei dati. Quando colleghiamo la nostra applicazione a Firebase, non ci stiamo connettendo attraverso una normale chiamata HTTP, ma ci stiamo connettendo attraverso una WebSocket. Esse sono molto più veloci di HTTP. Tutti i nostri dati verranno aggiornati automaticamente attraverso una singola WebSocket con una velocità dipendente dalla rete dell'utente. Quando un utente modifica dei dati, tutti gli utenti connessi ricevono i dati aggiornati quasi istantaneamente.
