\chapter{Seminario Schede SIM}
Le SIM sono intese per memorizzare in sicurezza il numero d'identità dell'abbonato mobile internazionale (IMSI) e la sua relativa chiave $K_i$. Sulla SIM è presente una CPU, una ROM per memorizzare il sistema operativo e le applicazioni, una EEPROM per la memoria (con cicli di scrittura/rimozione limitati), una RAM, un file system, il quale memorizza le chiavi di sessione, i contatti e gli SMS, una macchina virtuale Java e le relative applicazioni Java. Le applicazioni della SIM vengono gestite da un gestore delle applicazioni (spesso è il GSM). Esse possono mostrare delle semplici interfacce utente ai dispositivi, lanciare URLs, inviare SMSs, etc. Loro possono essere installate silenziosamente da remoto dall'operatore considerato. Le schede SIM utilizzano diversi meccanismi di sicurezza:
\begin{itemize}
    \item \textbf{Fisici}: gli attuali meccanismi non possono leggere il loro contenuto. Loro sono pensate come delle "scatole nere" con il quale possiamo comunicare solamente attraverso dei specifici pacchetti chiamati APDUs.
    \item \textbf{Digitali}: attraverso numeri PIN e PUC.
    \item \textbf{Crittografici}: prevedono autenticazioni basate su funzioni di hash crittografiche oppure comunicazioni sicure con i server OTA (encryption e/o signature).
    \item \textbf{Di separazione}: prevedono una separazione delle applicazioni (es. il sandboxing di Java).
\end{itemize}
Per installare nuove applicazioni gli operatori inviano degli SMS (OTA server $\leftrightarrow$ SIM), e per evitarne un uso improprio, vengono criptati oppure firmati. L'algoritmo che veniva comunemente utilizzato per firmare tali messaggi era DES e una chiave OTA. Le chiavi OTA e tutte le informazioni segrete generate sulla SIM sono basate sulla chiave $K_i$. Nel 2013 furono dimostrati alcuni punti deboli presenti su molte schede SIM. Infatti, inviando una richiesta di firma mal formata ad una SIM, in alcuni casi la SIM rispondeva con un messaggio di errore. La chiave OTA di una signature realizzata con l'algoritmo DES può essere individuata attraverso le tabella arcobaleno in un tempo ragionevole. Per questo motivo, molti operatori hanno deciso di aggiornare il loro meccanismo di comunicazione, iniziando ad utilizzare 3DES. Se viene utilizzata solamente una chiave, tale approccio è vulnerabile al \textbf{Downgrade attack}. Se ad alcune schede SIM viene richiesto di utilizzare un determinato meccanismo di sicurezza con un livello di sicurezza inferiore al loro livello di sicurezza di default, loro risponderanno declinando la richiesta firmando il messaggio con il meccanismo richiesto. Ad esempio, possiamo chiedere di utilizzare DES e otteniamo la prima delle tre chiavi. Poi possiamo chiedere di utilizzare 2DES ed otteniamo la seconda chiave. Infine, chiediamo di utilizzare 3DES, riuscendo a recuperare l'ultima chiave. Una volta che un virus Java viene caricato ha molto potenziale, ed attraverso l'escaping sandbox è possibile accedere a qualsiasi cosa sulla SIM da un'applicazione Java. Questo non include i dati presenti sul telefono, oppure l'accesso al sistema Android/iOS, visto che la SIM è una macchina separata. Per risolvere tale problema, erano disponibili diverse soluzioni :
\begin{itemize}
    \item disattivare OTA.
    \item utilizzare 3DES oppure le chiavi OTA.
    \item non inviare messaggi di risposta criptati a richieste mal formate.
\end{itemize}
Con il passare degli anni è stato sviluppato un nuovo attacco chiamato \textbf{Simjacker}, il quale sfrutta l'applicazione S@T Browser delle SIMs. Questo attacco coinvolge l'invio di un SMS in binario inviato ad un telefono cellulare con una scheda SIM vulnerabile. Questo SMS contiene un numero di istruzioni, le quali utilizzano un ambiente di esecuzione non sicuro presente sulla scheda SIM per eseguire la logica ed eseguire i comandi sulla SIM, e da tale ambiente al telefono stesso. L'attacco è costituito da due fasi :
\begin{enumerate}
    \item \textbf{Fase di attacco} : un SMS di attacco viene inviato dall'attaccante al numero di telefono della vittima. Tale SMS richiede le informazioni di localizzazione (id della cella di servizio e il codice IMEI del dispositivo) al dispositivo. Queste istruzioni rientrano in una serie di istruzioni di un insieme di strumenti della SIM (STK). L'attaccante riceverà le informazioni richieste tramite un secondo SMS.
    \item \textbf{Fase di esfiltrazione} : un SMS per i dati viene inviato dal dispositivo della vittima ad un dispositivo di destinazione. Questa attività non è osservabile dalla vittima, visto che non sono vengono riportate indicazioni sul dispositivo.
\end{enumerate}
Tale attacco per avere successo, deve avere bisogno che il dispositivo vittima sia in grado di ricevere SMS OTA e che la SIM utilizzi la tecnologia S@T Browser. Al browser può essere associato un livello minimo di sicurezza (MSL) tra cui : nessuna sicurezza e 3DES. Naturalmente, se non viene utilizzato alcun livello di sicurezza, qualsiasi persona può inviare dei messaggi al dispositivo vittima senza essere autenticato. Le variabili nell'ambiente S@T rappresentano dove le rispettive informazioni vengono memorizzate prima di essere inviate ad altri dispositivi. Nel caso dell'attacco Simjacker, le variabili sono temporanee. Questo significa che vengono eliminate quando : il S@T browser diventa inattivo, il S@T browser esegue una scheda con flag ResetVar settato negli attributi della scheda, un messaggio di alta priorità viene ricevuto. Un S@T browser diventa inattivo dopo che viene eseguito l'ultimo comando della scheda e non è stata effettuata alcuna ramificazione. E' stato osservato che gli attaccanti sono molto attenti ad eliminare le variabili temporanee. Infatti, in situazioni dove loro richiedono le informazioni di locazione in veloce successione (2 messaggi inviati allo stesso dispositivo nel giro di pochi secondi), loro specificano un alto livello di priorità per il secondo messaggio per assicurarsi che non vi è alcuna conservazione dei precedenti valori settati per il primo messaggio (messaggio specificato con bassa priorità). Viste le potenzialità di questo attacco, ne sono state realizzate diverse varianti (sono state provate circa 1000 combinazioni di differenti tipi di encoding nell'header dei pacchetti SMS e circa 860 varianti di tale attacco). Inoltre, sono presenti una varietà di differenti attacchi possibili basati sul S@T browser, che sfruttando diverse istruzioni rientrano nelle aree di studio delle applicazioni fraudolente, sistemi di tracciamento avanzati, assistenza nel rilascio dei malware e Denial of Service. Comunque, il S@T browser non è l'unica applicazione delle schede SIM che può essere sfruttata; in teoria qualsiasi applicazione delle SIM potrebbe essere soggetta a questo tipo di attacco (ad esempio Wireless Internet Browser). WIB è una tecnologia specificata da SmartTrust per la navigazione basata su schede SIM. La sua descrizione non è generalmente disponibile, ma in rete si trovano diversi documenti della sua implementazione da parte di altre aziende. A differenza del S@T browser, alcune documentazioni inerenti a WIB affermano almeno che il livello minimo di sicurezza "nessuna sicurezza" dovrebbe essere utilizzato solamente per le fasi di test, visto che non fornisce alcuna sicurezza (nonostante ciò, alcuni paesi utilizzano nessuna sicurezza OTA per WIB). Le potenziali mitigazioni sono simili al S@T browser. Dal lato della rete viene richiesto un filtro degli SMS per bloccare questi messaggi. Il miglioramento della sicurezza delle applicazioni S@T e WIB vittime, come il filtro per i messaggi in entrata e in uscita, non dipende solamente dal minimo livello di sicurezza per l'applicazione ma anche dalle configurazioni dell'applicazione.