\chapter{Cloud Computing}
Il \textbf{Cloud Computing} è un modo per utilizzare le infrastrutture IT senza il bisogno di installare uno specifico hardware relativo all'infrastruttura da utilizzare. L'infrastruttura IT può essere una macchina virtuale, una piattaforma software utilizzata per sviluppare ed eseguire applicazioni su diverse macchine, oppure un'applicazione software. Una delle caratteristiche principali del Cloud Computing è la capacità di fornire risorse astratte, oppure risorse virtualizzate fornite come un servizio e con una propria interfaccia. Le tipologie principali di CC sono:
\begin{itemize}
\item \textbf{IaaS}: Essi sono dei servizi di cloud computing che forniscono APIs di alto livello utilizzate per dereferenziare vari dettagli di basso livello dell'infrastruttura di rete considerata. Questi servizi sono sostenuti da data centers su larga scala costituiti da migliaia di computers. Tutte le istruzioni di un programma sono eseguiti all'interno di una macchina virtuale: questa fornisce isolamento, sicurezza, affidabilità e delle migliori prestazioni. Un'altra caratteristica fondamentale di tale infrastruttura è la mobilità dell'applicazione. Questo viene raggiunto incapsulando il sistema operativo ospite all'interno della macchina virtuale e consentendo di essere sospeso, completamente serializzato, migrato ad una piattaforma differente e ripristinato immediatamente oppure memorizzato per un futuro ripristino. Ci sono due tipologie di macchine virtuali: \textbf{native} ed \textbf{emulate}. Le macchine virtuali native sono costituite da una macchina logica che viene eseguita sulla stessa macchina fisica. Macchine logiche e fisiche possiedono lo stesso ISA e le istruzioni sulle macchine fisiche sono in gran parte eseguite su una CPU reale. Nella macchine virtuali emulate, le macchine fisiche e logiche sono differenti e possono avere ISA differenti. E' presente un hardware e un livello di linguaggio (Java) di emulazione.
\item \textbf{PaaS}: Essa è una piattaforma che offre un ambiente in cui gli sviluppatori creano e rilasciano le applicazioni. Essa supporta differenti linguaggi di programmazione (Java, Ruby) e databases. Gli utenti posso decidere la dimensione della macchina virtuale, la locazione, etc. e avere una console web per creare applicazioni. Loro possono utilizzare un IDE per sviluppare un'applicazione ed utilizzare un SDK o CLI per rilasciarla. I vantaggi principali di tale tipologia di CC sono: consente programmazione di alto livello riducendo la complessità, l'intero processo di sviluppo dell'applicazione è più efficiente visto che utilizza un'infrastruttura built-in, la manutenzione dell'applicazione è semplice. Alcuni svantaggi sono: gli sviluppatori potrebbero non essere in grado di utilizzare tutti gli strumenti convenzionali, ritrovandosi così bloccati in determinate piattaforme. Un esempio di PaaS è il motore delle app di Google; infatti, esso è un servizio cloud per eseguire applicazioni web nel data center di Google. Ha una trasparente scalabilità, semplice configurazione e supporta diversi linguaggi di programmazione. Le applicazione vengono eseguite all'interno di una sandbox per ragioni di sicurezza.
\item \textbf{SaaS}: SaaS è un modello di cloud computing nel quale un provider ospita delle applicazioni e le rende disponibili ai clienti in rete. Esso offre un elevata scalabilità e dei pagamenti flessibili, il quale consente ai clienti di accedere a dei servizi addizionali su richiesta. Molte applicazioni SaaS possono essere eseguite direttamente da un browser web, senza alcuni download o installazione richiesta, richiedono solamente alcuni plugins. Con SaaS è semplice per le aziende ottimizzare la loro manutenzione e il supporto, visto che ogni cosa può essere gestita dai fornitori: applicazioni, dati, middleware, virtualizzazione, etc. Un esempio di SaaS sono le applicazioni di Google.
\end{itemize}
